% Resumen
\begin{abstract}
Este trabajo presenta una arquitectura multimodal tipo \textit{Two-Tower} para la recomendación secuencial de música, diseñada para mitigar el problema de arranque en frío y enriquecer la representación latente de los ítems. El modelo integra cuatro modalidades: audio (espectrogramas Mel procesados con ResNet-18), texto (letras codificadas con mDeBERTa y LoRA), imágenes (carátulas procesadas con ResNet-18 preentrenada) y metadatos tabulares. Utilizando un dataset \textit{ad-hoc} de 10,000 canciones y 3 millones de interacciones de Last.fm, Spotify y YouTube, empleamos una estrategia de fusión tardía y una función de pérdida \textit{Triplet Loss} para optimizar el espacio métrico compartido. Los resultados experimentales muestran un Recall@10 de 0.6225 y un NDCG@10 de 0.5478, superando a los enfoques unimodales y validando la eficacia de la fusión de información heterogénea para capturar la semántica compleja de la experiencia musical.
\end{abstract}

\textbf{Palabras clave:} Sistemas de Recomendación, Aprendizaje Multimodal, Two-Tower, Fusión Tardía, Procesamiento de Audio, NLP, Visión por Computadora.
