% TRIZ
\section{Metodología TRIZ}

Se aplicó TRIZ para resolver la contradicción entre precisión del modelo y costo computacional.

\subsection{Contradicción y Principios}
Buscamos aumentar la complejidad semántica (multimodalidad) sin hacer inviable el entrenamiento. Se aplicaron los siguientes principios:
\begin{itemize}
    \item \textbf{Acción Preliminar}: pre-cómputo de espectrogramas y embeddings textuales para reducir carga en tiempo real.
    \item \textbf{Calidad Local}: codificadores especializados (ResNet, mDeBERTa) optimizados para cada modalidad.
    \item \textbf{Fusión}: estrategia de \textit{Late Fusion} para integrar representaciones de alto nivel.
    \item \textbf{Cambio de Parámetros}: uso de LoRA para adaptar modelos de lenguaje grandes con pocos recursos.
\end{itemize}

Esta metodología validó el diseño de una arquitectura eficiente y potente.
