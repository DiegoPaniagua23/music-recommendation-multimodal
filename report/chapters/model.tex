% Modelo
\section{Propuesta: Modelo Desarrollado}
\label{sec:modelo}

Se propone una arquitectura \textit{Two-Tower} híbrida que combina la codificación secuencial del historial del usuario mediante un Transformer (estilo SASRec) con una representación multimodal del ítem musical. Ambas torres proyectan sus entradas a un espacio latente común de dimensión $d=256$, optimizado mediante una función de pérdida contrastiva InfoNCE.

\subsection{Arquitectura del Modelo}

\subsubsection{Torre del Usuario (Sequential User Encoder)}
A diferencia de los enfoques tradicionales basados en RNNs, empleamos un codificador basado en Transformer para capturar dependencias a largo plazo en el historial de escucha.
\begin{itemize}
    \item \textbf{Entrada}: secuencia de IDs de canciones ($L=50$) más atributos demográficos (género, país).
    \item \textbf{Codificación}: se suman \textit{embeddings} de ítem y posicionales aprendibles. La secuencia pasa por un \textit{Transformer Encoder} de 2 capas y 4 cabezas de atención ($d_{model}=256$).
    \item \textbf{Fusión de Usuario}: la salida del Transformer (estado correspondiente al último ítem) se concatena con los \textit{embeddings} de género ($d=16$) y país ($d=32$), y se proyecta mediante un MLP a la dimensión final de 256.
\end{itemize}

\begin{figure}[htbp]
    \centering
    \resizebox{1.0\textwidth}{!}{\documentclass[tikz,border=10pt]{standalone}
\usepackage[utf8]{inputenc}
\usepackage{tikz}
\usetikzlibrary{shapes, arrows.meta, positioning, calc, shadows.blur, backgrounds, fit}

% Paleta de colores consistente
\definecolor{paperPurple}{RGB}{147, 112, 219} % MediumPurple para Sequence/Transformer
\definecolor{paperOrange}{RGB}{255, 165, 0}  % Orange para Demographics (Tabular)
\definecolor{paperGray}{RGB}{119, 136, 153}  % LightSlateGray para Fusion
\definecolor{paperBlue}{RGB}{70, 130, 180}   % SteelBlue para Embeddings básicos

\begin{document}

\begin{tikzpicture}[
    node distance=1.0cm and 1.2cm,
    font=\sffamily\normalsize,
    % Estilo base para bloques
    block/.style={
        draw, 
        rounded corners=2pt, 
        minimum height=1.5cm, 
        minimum width=3.8cm, 
        align=center,
        blur shadow={shadow blur steps=5},
        thick
    },
    % Estilo para inputs
    input_node/.style={
        align=right,
        font=\sffamily\bfseries\normalsize
    },
    % Estilo para flechas
    arrow/.style={
        -{Latex[length=3mm]}, 
        thick, 
        darkgray
    },
    % Operadores (Suma y Concat)
    operator/.style={
        circle, 
        draw, 
        thick, 
        fill=white, 
        inner sep=1pt, 
        minimum size=0.6cm
    }
]

    % --- 1. Rama Principal: Secuencia de Interacciones (SASRec) ---
    
    % Input Sequence
    \node[input_node] (in_seq) {
        History IDs\\
        \small $(B \times L)$
    };

    % Item Embedding
    \node[block, draw=paperBlue!80!black, fill=paperBlue!10, right=0.8cm of in_seq] (item_emb) {
        \textbf{Item Embedding}\\
        \small Lookup Table\\
        \small $dim=256$
    };
    \draw[arrow] (in_seq) -- (item_emb);

    % Position Embedding (Se suma, no es input externo)
    \node[block, draw=paperBlue!80!black, fill=paperBlue!10, below=0.5cm of item_emb] (pos_emb) {
        \textbf{Position Emb.}\\
        \small Learnable\\
        \small $dim=256$
    };

    % Suma (Add)
    \node[operator, label={180:\small Add}] (add_op) at ($(item_emb.east) + (1.0, -0.6)$) {$+$};


    % \node[circle, draw, thick, fill=white, inner sep=1pt, minimum size=0.6cm, label={180:\sffamily\bfseries\small Concat}] (concat) at (concat_point) {$\parallel$};
    
    % Conexiones a la suma
    \draw[arrow] (item_emb.east) -| (add_op);
    \draw[arrow] (pos_emb.east) -| (add_op);

    % Transformer Block (SASRec Core)
    \node[block, draw=paperPurple!80!black, fill=paperPurple!10, right=0.8cm of add_op, text width=4.2cm, minimum height=1.8cm] (transformer) {
        \textbf{Transformer Encoder}\\
        \small (SASRec Core)\\
        \small $L=2, H=4, D=256$\\
        \textit{\small + Causal Mask}
    };
    \draw[arrow] (add_op) -- (transformer);

    % Gather Last State (Extract h_L)
    \node[block, draw=paperPurple!80!black, fill=paperPurple!10, right=0.8cm of transformer] (gather) {
        \textbf{Gather Last}\\
        \small Select $h_t$ at $t=len$\\
        \small $(B \times 256)$
    };
    \draw[arrow] (transformer) -- (gather);


    % --- 2. Rama Demográfica (Inferior) ---
    
    % Inputs
    \node[input_node, below=2.5cm of in_seq] (in_gender) {User Gender\\ \small $(B \times 1)$};
    \node[input_node, below=0.5cm of in_gender] (in_country) {User Country\\ \small $(B \times 1)$};

    % Embeddings Demográficos
    \node[block, draw=paperOrange!80!black, fill=paperOrange!10, right=0.8cm of in_gender, minimum width=3.5cm] (gender_emb) {
        \textbf{Gender Emb.}\\
        \small $dim=16$
    };
    
    \node[block, draw=paperOrange!80!black, fill=paperOrange!10, right=0.8cm of in_country, minimum width=3.5cm] (country_emb) {
        \textbf{Country Emb.}\\
        \small $dim=32$
    };

    \draw[arrow] (in_gender) -- (gender_emb);
    \draw[arrow] (in_country) -- (country_emb);

    % --- 3. Concatenación y Fusión ---

    % Punto de concatenación
    % Alineado horizontalmente con 'gather' y verticalmente centrado entre las ramas
    \coordinate (concat_x) at ($(gather.east) + (1.5, 0)$);
    \coordinate (concat_y) at ($(gather.east)!0.5!(gender_emb.east)$); % Punto medio Y
    
    % Nodo Concat
    \node[operator, label={90:\sffamily\bfseries\small Concat}] (concat) at (concat_x |- gather) {$\parallel$};

    % Rutas hacia Concat
    \draw[arrow] (gather) -- (concat);
    
    % Rutas complejas para demográficos
    \draw[arrow] (gender_emb.east) -| (concat);
    \draw[arrow] (country_emb.east) -| ($(concat) + (0, -0.8)$) -- (concat);

    % Fusion Layer Final
    \node[block, draw=paperGray!80!black, fill=paperGray!10, right=0.8cm of concat, minimum height=2.8cm, text width=3.5cm] (fusion) {
        \textbf{Fusion Layer}\\
        \small Linear($256+16+32$)\\
        $\downarrow$\\
        \small LN + ReLU\\
        $\downarrow$\\
        \small Linear(256)
    };
    \draw[arrow] (concat) -- (fusion);

    % Output Final
    \node[right=0.8cm of fusion, font=\bfseries, align=center] (out_final) {
        User Embedding\\
        \small $(1 \times 256)$
    };
    \draw[arrow] (fusion) -- (out_final);


    % --- Fondo (Opcional) ---
    \begin{scope}[on background layer]
        \node[fit=(in_seq)(in_country)(out_final), fill=white, inner sep=0.5cm] {};
    \end{scope}

\end{tikzpicture}
\end{document}}
    \caption{Arquitectura de la Torre del Usuario. Se muestra el procesamiento secuencial con Transformer y la integración de atributos demográficos.}
    \label{fig:user_tower}
\end{figure}

\subsubsection{Torre del Ítem Multimodal}
Para capturar la naturaleza heterogénea de la música, se emplean codificadores especializados para cada modalidad, integrados mediante fusión tardía. Cada codificador proyecta su modalidad a un vector de dimensión $d=128$.
\begin{itemize}
    \item \textbf{Audio}: se procesan espectrogramas de Mel (1 canal) mediante una ResNet-18 modificada. A diferencia de la visión, no se utilizan pesos preentrenados, ya que los patrones espectrales difieren estructuralmente de las imágenes naturales.
    \item \textbf{Visual}: las carátulas de álbumes se codifican con una ResNet-18 inicializada con pesos de ImageNet, aprovechando la transferencia de aprendizaje para características estéticas.
    \item \textbf{Texto}: las letras y metadatos se procesan con mDeBERTa-v3-base. Se aplica LoRA (Rango=8, Alpha=32) a las proyecciones \textit{query} y \textit{value} para un ajuste fino eficiente, seguido de \textit{Mean Pooling}.
    \item \textbf{Tabular}: características numéricas y categóricas se procesan mediante un MLP de dos capas con \textit{Batch Normalization} y \textit{Dropout}.
\end{itemize}

\begin{figure}[htbp]
    \centering
    \resizebox{1.0\textwidth}{!}{\documentclass[tikz,border=10pt]{standalone}
\usepackage[utf8]{inputenc}
\usepackage{tikz}
\usetikzlibrary{shapes, arrows.meta, positioning, calc, shadows.blur, backgrounds, fit}

% Definición de colores profesionales (Palette estilo Paper)
\definecolor{paperBlue}{RGB}{70, 130, 180}   % SteelBlue para Audio
\definecolor{paperRed}{RGB}{205, 92, 92}     % IndianRed para Visual
\definecolor{paperGreen}{RGB}{60, 179, 113}  % MediumSeaGreen para Texto
\definecolor{paperOrange}{RGB}{255, 165, 0}  % Orange para Tabular
\definecolor{paperGray}{RGB}{119, 136, 153}  % LightSlateGray para Fusion
\definecolor{loraGold}{RGB}{218, 165, 32}    % GoldenRod para LoRA

\begin{document}

\begin{tikzpicture}[
    node distance=1.2cm and 1.5cm,
    font=\sffamily\normalsize,
    % Estilo base para bloques
    block/.style={
        draw, 
        rounded corners=2pt, 
        minimum height=1.6cm, 
        minimum width=4.5cm, 
        align=center,
        blur shadow={shadow blur steps=5},
        thick
    },
    % Estilo para inputs
    input_node/.style={
        align=right,
        font=\sffamily\bfseries\normalsize
    },
    % Estilo para flechas
    arrow/.style={
        -{Latex[length=3mm]}, 
        thick, 
        darkgray
    },
    % Estilo para el módulo LoRA
    lora/.style={
        draw=loraGold!80!black,
        fill=loraGold!10,
        dashed,
        rounded corners=2pt,
        minimum height=1.2cm,
        minimum width=3.5cm,
        align=center,
        thick
    }
]

    % --- 1. Definición de la Columna de Encoders (Backbones) ---
    
    % Audio Encoder (ResNet18 1-ch)
    \node[block, draw=paperBlue!80!black, fill=paperBlue!10] (audio_enc) {
        \textbf{Audio Encoder}\\
        \small ResNet18 (Mod. Conv1)\\
        \small $dim=128$
    };

    % Visual Encoder (ResNet18 ImageNet) - Debajo de Audio
    \node[block, draw=paperRed!80!black, fill=paperRed!10, below=of audio_enc] (visual_enc) {
        \textbf{Visual Encoder}\\
        \small ResNet18 (ImageNet Weights)\\
        \small $dim=128$
    };

    % Text Encoder (mDeBERTa) - Debajo de Visual
    \node[block, draw=paperGreen!80!black, fill=paperGreen!10, below=1.8cm of visual_enc] (text_enc) {
        \textbf{Text Encoder}\\
        \small mDeBERTa-v3-base\\
        \small (Projection $\rightarrow$ 128)
    };

    % Tabular Encoder (MLP) - Debajo de Texto
    \node[block, draw=paperOrange!80!black, fill=paperOrange!10, below=1.2cm of text_enc] (tab_enc) {
        \textbf{Tabular Encoder}\\
        \small MLP (BN + Dropout)\\
        \small $dim=128$
    };

    % --- 2. Módulo LoRA (Detalle Específico) ---
    % Dibujamos el LoRA "colgando" del Text Encoder como en tu dibujo
    \node[lora, above=0.2cm of text_enc] (lora_mod) {
        \textbf{LoRA Adapter}\\
        \small $r=8, \alpha=32$\\
        \small Targets: $q\_proj, v\_proj$
    };
    
    % Líneas conectando LoRA al backbone principal
    \draw[thick, loraGold!80!black] (text_enc.north west) ++(0.5,0) -- (lora_mod.south west);
    \draw[thick, loraGold!80!black] (text_enc.north east) ++(-0.5,0) -- (lora_mod.south east);

    % --- 3. Inputs (Izquierda) ---
    
    % Input Audio
    \node[input_node, left=1.5cm of audio_enc] (in_audio) {
        Mel Spectrograms\\
        \small $(1 \times 128 \times 128)$
    };
    \draw[arrow] (in_audio) -- (audio_enc);

    % Input Visual
    \node[input_node, left=1.5cm of visual_enc] (in_img) {
        Album Covers\\
        \small $(3 \times 224 \times 224)$
    };
    \draw[arrow] (in_img) -- (visual_enc);

    % Input Text
    \node[input_node, left=1.5cm of text_enc] (in_text) {
        Lyrics Tokens\\
        \small $(B \times 512)$
    };
    \draw[arrow] (in_text) -- (text_enc);

    % Input Tabular
    \node[input_node, left=1.5cm of tab_enc] (in_tab) {
        Tabular Features\\
        \small (Num + OneHot)
    };
    \draw[arrow] (in_tab) -- (tab_enc);

    % --- 4. Fusion Layer (Derecha) ---
    
    % Nodo de Concatenación (simbólico)
    \coordinate (concat_point) at ($(visual_enc.east)!0.5!(text_enc.east) + (2.5, 0)$);
    
    \node[circle, draw, thick, fill=white, inner sep=1pt, minimum size=0.6cm, label={180:\sffamily\bfseries\small Concat}] (concat) at (concat_point) {$\parallel$};
    
    % Flechas hacia la concatenación
    \draw[arrow] (audio_enc.east) -| (concat);
    \draw[arrow] (visual_enc.east) -- (concat_point |- visual_enc.east) -- (concat);
    \draw[arrow] (text_enc.east) -- (concat_point |- text_enc.east) -- (concat);
    \draw[arrow] (tab_enc.east) -| (concat);

    % Bloque de Fusión Final
    \node[block, draw=paperGray!80!black, fill=paperGray!10, right=0.8cm of concat, minimum height=3.5cm, text width=3.5cm] (fusion_layer) {
        \textbf{Fusion Layer}\\
        \small Linear(512)\\
        $\downarrow$\\
        \small BN + ReLU + Drop\\
        $\downarrow$\\
        \small Linear(256) + LN
    };

    \draw[arrow] (concat) -- (fusion_layer);

    % --- 5. Output Final ---
    \node[right=1cm of fusion_layer, font=\bfseries, align=center] (final_out) {Item Embedding\\ \small $(1 \times 256)$};
    \draw[arrow] (fusion_layer) -- (final_out);

    % --- Fondo y Labels Generales (Opcional) ---
    \begin{scope}[on background layer]
        \node[fit=(in_audio)(tab_enc)(final_out), fill=white, inner sep=0.5cm] {};
    \end{scope}

\end{tikzpicture}
\end{document}}
    \caption{Arquitectura de la Torre del Ítem Multimodal. Se ilustran los cuatro codificadores especializados y el módulo de fusión tardía.}
    \label{fig:item_tower}
\end{figure}

Fusión: los cuatro vectores de 128 dimensiones se concatenan ($d_{total}=512$) y pasan por un bloque de fusión (Linear $\rightarrow$ BN $\rightarrow$ ReLU $\rightarrow$ Dropout $\rightarrow$ Linear) que reduce la dimensión a 256, finalizando con una capa \textit{LayerNorm}.

\subsubsection{Modelo Two-Tower}
La arquitectura global integra el \textit{Sequential User Encoder} y el \textit{Multimodal Item Encoder} en un marco de aprendizaje contrastivo (ver Figura \ref{fig:two_towers_full}). El objetivo principal es alinear las representaciones de usuarios e ítems en un espacio latente compartido, donde la proximidad geométrica refleja la afinidad o probabilidad de interacción.

\paragraph{Espacio Latente y Normalización}
Ambas torres proyectan sus entradas a vectores de dimensión $d=256$. Un paso crucial en nuestra implementación es la normalización $L_2$ de los embeddings de salida, $\mathbf{u}$ y $\mathbf{i}$, antes del cálculo de similitud. Esto restringe los vectores a una hiperesfera unitaria, haciendo que el producto punto sea equivalente a la similitud coseno:
\begin{equation}
    \text{sim}(\mathbf{u}, \mathbf{i}) = \frac{\mathbf{u} \cdot \mathbf{i}}{\|\mathbf{u}\| \|\mathbf{i}\|} = \mathbf{u} \cdot \mathbf{i} \quad (\text{si } \|\mathbf{u}\|=\|\mathbf{i}\|=1)
\end{equation}

\subsection{Entrenamiento e Implementación}

\subsubsection{Función de Pérdida (InfoNCE)}
Para optimizar el espacio latente, utilizamos la función de pérdida InfoNCE (\textit{Noise Contrastive Estimation}), que maximiza la similitud entre pares positivos (usuario, ítem interactuado) y la minimiza con respecto a pares negativos (otros ítems en el mismo lote). Dado un lote de $N$ pares positivos $\{(\mathbf{u}_k, \mathbf{i}_k)\}_{k=1}^N$, la pérdida para el $k$-ésimo par se define como:
\begin{equation}
    \mathcal{L}_k = -\log \frac{\exp(\text{sim}(\mathbf{u}_k, \mathbf{i}_k) / \tau)}{\sum_{j=1}^N \exp(\text{sim}(\mathbf{u}_k, \mathbf{i}_j) / \tau)}
\end{equation}
donde $\tau$ es un hiperparámetro de temperatura (fijado en 0.07) que controla la suavidad de la distribución de probabilidad. Esta formulación permite un entrenamiento eficiente utilizando los otros elementos del lote como negativos implícitos (\textit{in-batch negatives}).

\subsubsection{Configuración Experimental}
El modelo fue implementado en PyTorch y entrenado en un clúster de computación de alto rendimiento (HPC) utilizando la biblioteca \texttt{uv} para la gestión de dependencias y entornos.
\begin{itemize}
    \item \textbf{Hardware}: Entrenamiento distribuido en múltiples GPUs (Distributed Data Parallel - DDP) para acelerar el proceso y manejar lotes más grandes.
    \item \textbf{Hiperparámetros}:
    \begin{itemize}
        \item \textbf{Optimizador}: AdamW con una tasa de aprendizaje inicial de $1 \times 10^{-4}$.
        \item \textbf{Batch Size}: 64 por GPU.
        \item \textbf{Epochs}: 10 épocas completas.
        \item \textbf{Mixed Precision}: Se utilizó precisión mixta automática (AMP) para reducir el uso de memoria y acelerar el cómputo.
    \end{itemize}
    \item \textbf{Estrategia de Validación}: Se empleó una métrica de Recall@10 calculada sobre el conjunto de validación al final de cada época para monitorear el rendimiento y guardar el mejor modelo (\textit{checkpointing}).
\end{itemize}

\paragraph{Función de Pérdida InfoNCE Simétrica}
Para el entrenamiento, utilizamos la función de pérdida InfoNCE (Information Noise Contrastive Estimation), adaptada para considerar la simetría entre usuarios e ítems (similar a CLIP). Dado un lote de tamaño $B$, calculamos la matriz de logits escalada por una temperatura aprendible $\tau$ (inicializada en 0.07):
\begin{equation}
    \text{logits} = \frac{\mathbf{U} \mathbf{I}^T}{\tau}
\end{equation}
La pérdida total es el promedio de la pérdida usuario-a-ítem ($\mathcal{L}_{u2i}$) y la pérdida ítem-a-usuario ($\mathcal{L}_{i2u}$), lo que maximiza la similitud de los pares positivos (diagonal) y minimiza la de los negativos (fuera de la diagonal) en ambas direcciones. Además, implementamos un enmascaramiento de colisiones para evitar penalizar falsos negativos cuando un mismo usuario aparece múltiples veces en el mismo lote.

\begin{figure}[h]
    \centering
    \resizebox{0.95\textwidth}{!}{\documentclass[tikz, border=20pt]{standalone}
\usepackage[utf8]{inputenc}
\usepackage{tikz}
\usetikzlibrary{positioning, fit, calc, shapes.geometric, arrows.meta, shadows.blur, backgrounds}
\usepackage{fontawesome5} 
\usepackage{helvet}
\renewcommand{\familydefault}{\sfdefault}

% --- Paleta de Colores ---
\definecolor{userPurple}{RGB}{147, 112, 219} 
\definecolor{itemBlue}{RGB}{70, 130, 180}    
\definecolor{itemRed}{RGB}{205, 92, 92}      
\definecolor{itemGreen}{RGB}{60, 179, 113}   
\definecolor{itemOrange}{RGB}{255, 165, 0}   
\definecolor{fusionGray}{RGB}{119, 136, 153} 
\definecolor{loraGold}{RGB}{218, 165, 32}    
\definecolor{bgGray}{RGB}{250, 250, 250}     

\begin{document}

\begin{tikzpicture}[
    node distance=1.2cm and 1.5cm,
    font=\sffamily\small,
    % Estilos
    block/.style={
        rectangle, 
        draw=black!60, 
        fill=white, 
        thick, 
        rounded corners=4pt, 
        minimum height=1.2cm, 
        text width=3.8cm, 
        align=center,
        blur shadow={shadow blur steps=3, shadow opacity=20}
    },
    input_node/.style={
        rectangle,
        rounded corners=6pt,
        draw=black!40,
        fill=white,
        thick,
        minimum height=0.9cm,
        text width=3.0cm,
        align=center,
        font=\sffamily\footnotesize
    },
    fusion/.style={
        block,
        fill=fusionGray!15,
        draw=fusionGray!80!black,
        minimum height=1cm,
        font=\sffamily\bfseries\small
    },
    operator/.style={
        circle, 
        draw, 
        thick, 
        fill=white, 
        inner sep=0pt, 
        minimum size=0.8cm,
        font=\large\bfseries
    },
    arrow/.style={
        -{Stealth[length=3mm, width=2mm]}, 
        thick, 
        draw=black!70,
        rounded corners=5pt
    }
]

    % =========================================================
    % 1. TOP SECTION (SALIDA)
    % =========================================================
    
    % AJUSTE 1: Etiqueta "Dot Product" movida ABAJO (below)
    % Esto libera completamente el espacio de arriba para la flecha de salida
    \node[operator, label={below:\textbf{Dot Product}}] (dot_prod) at (5.5, 13.5) {$\otimes$};
    
    \node[block, above=0.8cm of dot_prod, text width=3.5cm, fill=black!5, dashed] (output_prob) {
        \textbf{Similarity Score}\\
        \scriptsize Sigmoid($u^T v$)
    };
    \draw[arrow] (dot_prod) -- (output_prob);

    % =========================================================
    % 2. USER TOWER (Izquierda)
    % =========================================================
    
    \node[fusion] (user_fusion) at (0, 9.5) {
        User Fusion Layer\\
        \scriptsize Linear $\rightarrow$ LN $\rightarrow$ ReLU
    };

    \node[operator, below=0.8cm of user_fusion, label={135:\scriptsize Concat}] (user_concat) {$\parallel$};

    % --- Transformer ---
    \node[block, below=1.5cm of user_concat, xshift=-1.5cm, fill=userPurple!10, draw=userPurple!80!black] (transformer) {
        \textbf{Transformer Encoder}\\
        \scriptsize (SASRec Core)\\
        \scriptsize $L=2, H=4$
    };
    
    \node[block, below=0.6cm of transformer, fill=userPurple!05, draw=userPurple!50, minimum height=0.8cm] (pos_emb) {
        \scriptsize Positional Emb. $\oplus$ Item Emb.
    };
    
    \node[input_node, below=0.8cm of pos_emb] (in_hist) {
        \faHistory \ \textbf{History IDs}
    };

    % --- Demographics ---
    \node[block, right=0.8cm of transformer, fill=itemOrange!10, draw=itemOrange!80!black, text width=2.8cm] (demo_emb) {
        \textbf{Demographics}\\
        \scriptsize Gender + Country\\
        \scriptsize (Embeddings)
    };
    
    \node[input_node] (in_demo) at (demo_emb |- in_hist) {
        \faUser \ \textbf{User Profile}
    };

    % Conexiones User Tower
    \draw[arrow] (in_hist) -- (pos_emb);
    \draw[arrow] (pos_emb) -- (transformer);
    \draw[arrow] (in_demo) -- (demo_emb);
    
    \draw[arrow] (transformer.north) |- (user_concat);
    \draw[arrow] (demo_emb.north) |- (user_concat);
    \draw[arrow] (user_concat) -- (user_fusion);

    % =========================================================
    % 3. ITEM TOWER (Derecha)
    % =========================================================
    
    \node[fusion] (item_fusion) at (11, 9.5) {
        Item Fusion Layer\\
        \scriptsize Linear $\rightarrow$ BN $\rightarrow$ Drop
    };

    \node[operator, below=0.8cm of item_fusion, label={180:\scriptsize Concat}] (item_concat) {$\parallel$};

    % --- Stack de Encoders ---
    \node[block, below=1.5cm of item_concat, fill=itemGreen!10, draw=itemGreen!80!black] (text_enc) {
        \textbf{Text Encoder}\\
        \scriptsize mDeBERTa-v3
    };
    \node[draw=loraGold, fill=loraGold!20, rounded corners=2pt, inner sep=3pt, font=\tiny\bfseries, anchor=north east, yshift=2pt] at (text_enc.north east) {LoRA};

    \node[block, below=0.6cm of text_enc, fill=itemRed!10, draw=itemRed!80!black] (visual_enc) {
        \textbf{Visual Encoder}\\
        \scriptsize ResNet18 (ImageNet)
    };

    \node[block, below=0.6cm of visual_enc, fill=itemBlue!10, draw=itemBlue!80!black] (audio_enc) {
        \textbf{Audio Encoder}\\
        \scriptsize ResNet18 (Spectrogram)
    };
    
    \node[block, below=0.6cm of audio_enc, fill=itemOrange!10, draw=itemOrange!80!black] (tab_enc) {
        \textbf{Tabular Encoder}\\
        \scriptsize MLP (Features)
    };

    % --- Input Unificado ---
    \node[input_node, below=1.2cm of tab_enc, text width=5.5cm] (in_item_data) {
        \faBoxOpen \ \textbf{Item Raw Data}\\
        \scriptsize (Lyrics, Cover, Audio, Metadata)
    };

    % --- Conexiones Entrada ---
    \draw[arrow] (in_item_data.north) -- (tab_enc.south);
    \draw[arrow] (in_item_data.west) -- ++(-0.5, 0) |- (audio_enc.west);
    \draw[arrow] (in_item_data.west) -- ++(-0.5, 0) |- (visual_enc.west);
    \draw[arrow] (in_item_data.west) -- ++(-0.5, 0) |- (text_enc.west);
    
    % --- Conexiones Salida (Bus Derecho) ---
    \draw[arrow] (text_enc.east) -- ++(0.4, 0) |- (item_concat);
    \draw[arrow] (visual_enc.east) -- ++(0.8, 0) |- (item_concat);
    \draw[arrow] (audio_enc.east) -- ++(1.2, 0) |- (item_concat);
    \draw[arrow] (tab_enc.east) -- ++(1.6, 0) |- (item_concat);
    
    \draw[arrow] (item_concat) -- (item_fusion);

    % =========================================================
    % 4. CONEXIONES FINALES PERFECTAS
    % =========================================================
    
    % AJUSTE 2: Usamos '|-' (Vertical-Horizontal) automático.
    % 'pos=0.75' pone la etiqueta en el tramo horizontal final.
    % 'above=3pt' asegura que no toque la línea.
    
    % User Side (Left)
    \draw[arrow] (user_fusion.north) |- (dot_prod.west) node[pos=0.75, above=3pt, font=\bfseries] {$u$};
    
    % Item Side (Right)
    \draw[arrow] (item_fusion.north) |- (dot_prod.east) node[pos=0.75, above=3pt, font=\bfseries] {$v$};

    % =========================================================
    % 5. BACKGROUNDS
    % =========================================================
    
    \begin{scope}[on background layer]
        \node[fit=(user_fusion)(in_hist)(in_demo)(demo_emb)(transformer), fill=bgGray, rounded corners=20pt, draw=black!10, inner sep=35pt, label={[anchor=north west, inner sep=15pt, font=\bfseries\color{gray!80}]north west:User Tower}] (user_bg) {};
        
        \coordinate (aux_right) at ($(text_enc.east)+(1.8,0)$); 
        \coordinate (aux_left) at ($(text_enc.west)+(-1.2,0)$); 
        
        \node[fit=(item_fusion)(tab_enc)(in_item_data)(aux_right)(aux_left), fill=bgGray, rounded corners=20pt, draw=black!10, inner sep=35pt, label={[anchor=north east, inner sep=15pt, font=\bfseries\color{gray!80}]north east:Item Tower}] (item_bg) {};
    \end{scope}

\end{tikzpicture}
\end{document}}
    \caption{Arquitectura General Two-Tower Multimodal. La arquitectura integra una Torre de Usuario (izquierda) que modela preferencias secuenciales mediante un Transformer, y una Torre de Ítem (derecha) que fusiona representaciones de audio, texto, imagen y metadatos. Ambas proyecciones se alinean en un espacio latente compartido optimizado mediante InfoNCE.}
    \label{fig:two_towers_full}
\end{figure}
%\clearpage

\subsection{Entrenamiento e Implementación}

El modelo se entrena optimizando la función de pérdida InfoNCE con temperatura $\tau=0.07$, que maximiza la similitud coseno entre el usuario y el ítem positivo mientras la minimiza frente a los otros ítems del lote (\textit{in-batch negatives}).
\begin{equation}
\mathcal{L} = -\log \frac{\exp(\text{sim}(\mathbf{u}, \mathbf{i}^+) / \tau)}{\sum_{j=1}^{B} \exp(\text{sim}(\mathbf{u}, \mathbf{i}_j) / \tau)}
\end{equation}
La implementación se realizó en \texttt{PyTorch} con soporte para entrenamiento distribuido (DDP) y precisión mixta (AMP). Se utilizó el optimizador AdamW ($lr=1e-4$, $batch\_size=64$) durante 10 épocas.
