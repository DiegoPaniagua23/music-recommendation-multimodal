\documentclass{ceurart}

%% One can fix some overfulls
\sloppy

%% Minted listings support 
%% Need pygment <http://pygments.org/> <http://pypi.python.org/pypi/Pygments>
\usepackage{listings}
%% auto break lines
\lstset{breaklines=true}

\usepackage{float}
\usepackage{booktabs}
\usepackage{array}
\usepackage{amsmath}
\usepackage{standalone}
\usepackage{tikz}
\usepackage[most]{tcolorbox}
\usepackage{fontawesome5}
\usetikzlibrary{shapes, arrows.meta, positioning, calc, shadows.blur, backgrounds, fit}

% Definición de colores para item_tower.tex
\definecolor{paperBlue}{RGB}{70, 130, 180}   % SteelBlue para Audio
\definecolor{paperRed}{RGB}{205, 92, 92}     % IndianRed para Visual
\definecolor{paperGreen}{RGB}{60, 179, 113}  % MediumSeaGreen para Texto
\definecolor{paperOrange}{RGB}{255, 165, 0}  % Orange para Tabular
\definecolor{paperGray}{RGB}{119, 136, 153}  % LightSlateGray para Fusion
\definecolor{loraGold}{RGB}{218, 165, 32}    % GoldenRod para LoRA
\definecolor{paperPurple}{RGB}{147, 112, 219} % MediumPurple para Sequence/Transformer

% Definición de colores para two-towers.tex
\definecolor{userPurple}{RGB}{147, 112, 219} 
\definecolor{itemBlue}{RGB}{70, 130, 180}    
\definecolor{itemRed}{RGB}{205, 92, 92}      
\definecolor{itemGreen}{RGB}{60, 179, 113}   
\definecolor{itemOrange}{RGB}{255, 165, 0}   
\definecolor{fusionGray}{RGB}{119, 136, 153} 
\definecolor{bgGray}{RGB}{250, 250, 250}     

\begin{document}

\copyrightyear{2025}
\copyrightclause{Copyright for this paper by its authors.
  Use permitted under Creative Commons License Attribution 4.0
  International (CC BY 4.0).}



\title{Arquitectura Two-Tower Multimodal para Recomendación Musical Secuencial: Fusión de Audio, Texto e Imagen mediante Aprendizaje Contrastivo (InfoNCE)}

\author{César Aguirre-Calzadilla}[
email=cesar.aguirre@cimat.mx,
]\fnmark[1]

\author{Gustavo Hernández-Angeles}[
email=gustavo.hernandez@cimat.mx,
]\fnmark[1]

\author{Diego Paniagua-Molina}[
email=diego.paniagua@cimat.mx,
]\fnmark[1]

\address{Mathematics Research Center (CIMAT—Centro de Investigación en Matemáticas), Graduate Program in Statistical Computing, Nuevo León, Mexico}

\fntext[1]{Estos autores contribuyeron de igual manera.}

\begin{abstract}
Este trabajo presenta una arquitectura multimodal tipo \textit{Two-Tower} para la recomendación secuencial de música, diseñada para mitigar el problema de arranque en frío y enriquecer la representación latente de los ítems. El modelo integra cuatro modalidades: audio (espectrogramas Mel procesados con ResNet-18), texto (letras codificadas con mDeBERTa y LoRA), imágenes (carátulas procesadas con ResNet-18 preentrenada) y metadatos tabulares. Utilizando un dataset \textit{ad-hoc} de 10,000 canciones y 3 millones de interacciones de Last.fm, Spotify y YouTube, empleamos una estrategia de fusión tardía y una función de pérdida \textit{Triplet Loss} para optimizar el espacio métrico compartido. Los resultados experimentales muestran un Recall@10 de 0.6225 y un NDCG@10 de 0.5478, superando a los enfoques unimodales y validando la eficacia de la fusión de información heterogénea para capturar la semántica compleja de la experiencia musical.
\end{abstract}

\begin{keywords}
  Sistemas de Recomendación \sep Aprendizaje Multimodal \sep Two-Tower \sep Fusión Tardía \sep Procesamiento de Audio \sep NLP \sep Visión por Computadora
\end{keywords}

\conference{}
\maketitle

\input{Chapters/introduccion}
% Planeación
\section{Gestión y Planeación}

\subsection{Matriz RACI}
Se asignaron responsabilidades específicas a los integrantes del equipo: César Aguirre (C), Gustavo Hernández (G) y Diego Paniagua (D), para optimizar la colaboración y asegurar el cumplimiento de los objetivos del proyecto. Cada integrante asumió roles de liderazgo en diferentes etapas, manteniendo una comunicación constante. La matriz de asignación de responsabilidades (RACI) se detalla en la Tabla \ref{tab:raci}.

\begin{table}[htbp]
\caption{Matriz RACI del Proyecto (R: Responsable, A: Aprobador, C: Consultado, I: Informado)}
\label{tab:raci}
\centering
\begin{tabular}{|l|c|c|c|}
\hline
\textbf{Actividad} & \textbf{César (C)} & \textbf{Gustavo (G)} & \textbf{Diego (D)} \\ \hline
Recolección de Datos (APIs) & R & C & C \\ \hline
Preprocesamiento Multimodal & C & R & I \\ \hline
Diseño de Arquitectura Two-Tower & C & C & R \\ \hline
Implementación del Modelo (InfoNCE) & R & C & I \\ \hline
Entrenamiento y Ajuste & I & R & C \\ \hline
Evaluación de Métricas & C & I & R \\ \hline
Redacción del Reporte Técnico & R/A & R/A & R/A \\ \hline
\end{tabular}
\end{table}

\subsection{Estructura de Desglose del Trabajo (WBS)}
\begin{itemize}
    \item \textbf{1. Datos}: extracción (Last.fm, YouTube, Genius, Spotify) y preprocesamiento (Mel-spectrograms, embeddings).
    \item \textbf{2. Modelado}: codificadores unimodales (ResNet, mDeBERTa), arquitectura Two-Tower y Late Fusion.
    \item \textbf{3. Evaluación}: métricas (Recall@k, NDCG@k) y experimentos.
    \item \textbf{4. Entrega}: reporte y código.
\end{itemize}

\subsection{Ruta Crítica}
Las tareas críticas fueron: 1) sincronización de datos multimodales, 2) pre-cómputo de características, 3) entrenamiento del modelo Two-Tower con fusión tardía.

\subsection{Infraestructura y Flujo de Trabajo}
El flujo de trabajo se estructura en tres pilares fundamentales para garantizar la reproducibilidad y la colaboración eficiente: gestión de dependencias, control de versiones de código y control de versiones de datos.

\subsubsection{Gestión de Dependencias: uv}
Para garantizar la consistencia entre los entornos de desarrollo, se utiliza \textbf{uv} como gestor de paquetes. Esta herramienta, escrita en Rust, destaca por su velocidad y permite asegurar que todos los miembros del equipo utilicen exactamente las mismas versiones de las librerías, eliminando conflictos de compatibilidad y gestionando la versión de Python del proyecto automáticamente.

\subsubsection{Control de Versiones: Git y GitHub}
El manejo del código fuente se realiza mediante una estrategia de ramificación \textit{Gitflow} simplificada, integrando un motor local (Git) y una plataforma en la nube (GitHub).
\begin{itemize}
    \item \textbf{Estrategia de Ramas}: se utiliza \texttt{main} para producción, \texttt{develop} para integración y ramas \texttt{feat/...} para el desarrollo de nuevas características.
    \item \textbf{Alcance}: Git gestiona exclusivamente archivos ligeros de código y configuración (\texttt{.py}, \texttt{.md}, \texttt{.yaml}, \texttt{.ipynb}).
\end{itemize}

\subsubsection{Gestión de Datos: DVC}
Dado que Git no es adecuado para archivos binarios pesados, se implementó \textbf{DVC (Data Version Control)} con almacenamiento remoto en Google Drive.
\begin{itemize}
    \item \textbf{Almacenamiento Híbrido}: Git almacena punteros ligeros (\texttt{.dvc}), mientras que los archivos reales (audios, imágenes) residen en la nube.
    \item \textbf{Reproducibilidad}: vincula versiones exactas del código con versiones exactas de los datos, permitiendo replicar experimentos con precisión.
\end{itemize}

% TRIZ
\section{Metodología TRIZ}

Se aplicó TRIZ para resolver la contradicción entre precisión del modelo y costo computacional.

\subsection{Contradicción y Principios}
Buscamos aumentar la complejidad semántica (multimodalidad) sin hacer inviable el entrenamiento. Se aplicaron los siguientes principios:
\begin{itemize}
    \item \textbf{Acción Preliminar}: pre-cómputo de espectrogramas y embeddings textuales para reducir carga en tiempo real.
    \item \textbf{Calidad Local}: codificadores especializados (ResNet, mDeBERTa) optimizados para cada modalidad.
    \item \textbf{Fusión}: estrategia de \textit{Late Fusion} para integrar representaciones de alto nivel.
    \item \textbf{Cambio de Parámetros}: uso de LoRA para adaptar modelos de lenguaje grandes con pocos recursos.
\end{itemize}

Esta metodología validó el diseño de una arquitectura eficiente y potente.

% Trabajos Relacionados
\section{Trabajos Relacionados}
\label{sec:relacionados}

El desarrollo de sistemas de recomendación efectivos requiere abordar tanto la naturaleza dinámica de las preferencias de los usuarios como la riqueza de información contenida en los ítems. En esta sección, analizamos la literatura existente en torno a dos pilares fundamentales: los modelos secuenciales y el aprendizaje multimodal. Además, contrastamos los enfoques clásicos con las técnicas de aprendizaje profundo y destacamos cómo nuestra propuesta integra estos avances en una arquitectura unificada para superar las limitaciones de los trabajos previos.

\subsection{Modelos Secuenciales}
\citet{hidasi2015session} introdujeron RNNs para recomendaciones basadas en sesiones, capturando dinámicas temporales pero ignorando el contenido, lo que limita su desempeño en \textit{cold-start}.

\subsection{Aprendizaje Multimodal}
\citet{oramas2018multimodal} combinaron texto, audio e imagen mediante fusión temprana, superando modelos unimodales pero sin mecanismos de atención. \citet{won2020multimodal} utilizaron aprendizaje métrico con redes siamesas para recuperación basada en etiquetas, destacando la importancia de espacios latentes compartidos.

\subsection{Enfoques Clásicos vs. Deep Learning}
\citet{murauer2018detecting} mostraron la eficacia de XGBoost con características artesanales para clasificación de géneros. En contraste, nuestra propuesta utiliza Deep Learning (ResNet-18) para extraer características jerárquicas automáticamente.

\subsection{Diferenciación de la Propuesta}
A diferencia de los trabajos previos, este proyecto propone una arquitectura \textit{Two-Tower} que:
\begin{itemize}
    \item Integra cuatro modalidades (audio, texto, imagen, tabular) mediante \textit{Late Fusion}.
    \item Combina codificación secuencial del usuario con representaciones ricas de contenido.
    \item Utiliza técnicas eficientes como LoRA para texto.
\end{itemize}


% Dataset
\section{Base de Datos y Recolección}
\label{sec:dataset}

Para este trabajo, se construyó un dataset multimodal \textit{ad-hoc} que integra información de audio, texto, imágenes y metadatos tabulares, alineados con un historial de interacciones de usuarios. El conjunto de datos base consta de 10,000 canciones únicas y más de 3 millones de interacciones provenientes de aproximadamente 850 usuarios de la plataforma Last.fm. A continuación, se detalla el proceso de adquisición y procesamiento para cada modalidad.

\subsection{Interacciones y Metadatos (Last.fm y Spotify)}
Los datos de interacción usuario-ítem se obtuvieron de Last.fm, recopilando eventos de reproducción que incluyen metadatos del usuario (género, país de residencia) y detalles de la reproducción (tiempo, canción). Complementariamente, se utilizaron identificadores de pistas (\textit{Track IDs}) de un dataset de Spotify disponible en Kaggle para enriquecer cada ítem con características tabulares de alto nivel.
\begin{itemize}
    \item \textbf{Features Numéricos}: Se seleccionaron 14 atributos acústicos proporcionados por la API de Spotify, incluyendo \textit{danceability}, \textit{energy}, \textit{valence}, \textit{tempo}, \textit{loudness}, entre otros. Estos valores fueron normalizados utilizando \textit{StandardScaler} para tener media cero y varianza unitaria.
    \item \textbf{Features Categóricos}: El género musical (\textit{track\_genre}) fue codificado utilizando \textit{One-Hot Encoding}, permitiendo al modelo capturar explícitamente la categoría estilística de la canción.
\end{itemize}

\subsection{Audio (YouTube)}
La recolección de los archivos de audio se realizó mediante un script personalizado que utiliza la herramienta \texttt{yt-dlp}.
\begin{itemize}
    \item \textbf{Adquisición}: se generaron búsquedas optimizadas en YouTube utilizando los metadatos limpios (artista y título). El script fue configurado para priorizar videos etiquetados como `Audio' o de alta calidad (\texttt{bestaudio}). Dado el volumen de descargas, se emplearon VPNs rotativas para mitigar bloqueos por actividad automatizada.
    \item \textbf{Preprocesamiento}:
    \begin{itemize}
        \item \textbf{Recorte Temporal (Windowing)}: se extrajo un segmento de 30 segundos por canción, específicamente del intervalo 00:30 a 01:00, para capturar la estructura representativa del tema (generalmente el coro o verso principal) y evitar introducciones silenciosas o irrelevantes.
        \item \textbf{Remuestreo (Downsampling)}: los audios originales de 44.1 kHz fueron remuestreados a 22.05 kHz. Siguiendo el Teorema de Nyquist, esta frecuencia es suficiente para representar componentes espectrales de hasta 11 kHz, donde reside la mayor parte de la información tímbrica distintiva de los géneros musicales.
        \item \textbf{Mezcla a Mono}: se convirtieron los canales estéreo a un solo canal mono para simplificar la entrada al modelo.
        \item \textbf{Generación de Espectrogramas Mel}: Se calcularon espectrogramas Mel utilizando una ventana FFT de 2048 muestras, un \textit{hop length} de 512 muestras y 128 bandas de frecuencia Mel. Los valores de potencia se convirtieron a escala de decibeles (dB) y se normalizaron al rango $[0, 1]$ utilizando escalado Min-Max. El resultado es un tensor de dimensión $(1, 128, 128)$ que representa visualmente el contenido espectral del audio.
    \end{itemize}
\end{itemize}

\subsection{Texto (Genius)}
Las letras de las canciones (\textit{lyrics}) se obtuvieron mediante la API de Genius. Se implementó una estrategia de búsqueda en cascada para maximizar la coincidencia.
\begin{itemize}
    \item \textbf{Cobertura}: se logró recuperar la letra para aproximadamente 8,500 canciones (85\% del dataset).
    \item \textbf{Datos Faltantes}: el 15\% restante corresponde principalmente a música instrumental, bandas sonoras (e.g., Hans Zimmer) o piezas clásicas que carecen de contenido lírico. Para estos casos, se utilizó un token especial de `vacío' en el modelo de lenguaje.
    \item \textbf{Procesamiento}: El texto crudo se tokeniza dinámicamente utilizando el tokenizador de \texttt{mDeBERTa-v3-base}, truncando las secuencias a una longitud máxima compatible con el modelo.
\end{itemize}

\subsection{Imágenes (Spotify)}
Utilizando la librería \texttt{Spotipy} y los \textit{Track IDs}, se consultó la API de Spotify para obtener las carátulas de los álbumes. La API proporciona imágenes en tres resoluciones (640x640, 300x300, 64x64). Se seleccionó la resolución de 300x300 píxeles.
\begin{itemize}
    \item \textbf{Transformaciones}: Durante el entrenamiento, las imágenes se redimensionan a $224 \times 224$ píxeles y se normalizan utilizando la media y desviación estándar del dataset ImageNet ($\mu=[0.485, 0.456, 0.406]$, $\sigma=[0.229, 0.224, 0.225]$), requisito para utilizar la red ResNet-18 preentrenada.
\end{itemize}

\begin{table}[htbp]
\centering
\caption{Resumen del dataset multimodal construido.}
\begin{tabular}{c c m{6cm}}
\toprule
\textbf{Modalidad} & \textbf{Fuente} & \textbf{Detalles Técnicos} \\
\midrule
Interacciones & Last.fm & $>$3M eventos, ~850 usuarios. \\
Audio         & YouTube & Clips 30s, Mel-Spectrogram (128x128). \\
Texto         & Genius & ~8.5K letras, Tokenización mDeBERTa. \\
Imagen        & Spotify & Portadas 300x300 $\rightarrow$ 224x224 (ImageNet Norm). \\
Tabular       & Spotify & 14 features numéricos + Género (One-Hot). \\
\bottomrule
\end{tabular}
\end{table}

\input{Chapters/modelo}
% Resultados
\section{Resultados}
\label{sec:resultados}

En esta sección se presentan los resultados obtenidos durante la evaluación del modelo propuesto. Las métricas utilizadas para medir el desempeño incluyen Recall@$k$ y NDCG@$k$, donde $k$ representa el número de ítems recomendados considerados.

\subsection{Resultados Globales}

Los resultados globales del modelo se resumen en la Tabla~\ref{tab:resultados-globales}.

\begin{table}[htbp]
    \centering
    \caption{Resultados globales de evaluación del modelo.}
    \label{tab:resultados-globales}
    \begin{tabular}{@{}cccc@{}}
        \toprule
        \textbf{Métrica} & \textbf{@10} & \textbf{@20} & \textbf{@50} \\
        \midrule
        Recall & 0.6225 & 0.6474 & 0.6757 \\
        NDCG   & 0.5478 & 0.5541 & 0.5597 \\
        \bottomrule
    \end{tabular}
\end{table}

Los valores obtenidos indican que el modelo logra un desempeño competitivo, especialmente en términos de Recall y NDCG, lo que sugiere que las recomendaciones generadas son relevantes y están bien ordenadas en las primeras posiciones.

\input{analisis_cualitativo}\subsection{Definición y Justificación de Métricas}

Para interpretar adecuadamente los resultados obtenidos, es fundamental comprender las métricas utilizadas en la evaluación del modelo: Recall y NDCG. Estas métricas son estándar en la industria y la academia para evaluar sistemas de ranking y recuperación de información.

\paragraph{Recall} Esta métrica mide la proporción de ítems relevantes que el sistema de recomendación logra recuperar dentro de un conjunto de $k$ recomendaciones. Matemáticamente, se define como:
\begin{equation}
    \text{Recall@k} = \frac{|\text{Ítems relevantes} \cap \text{Ítems recomendados}@k|}{|\text{Ítems relevantes}|}.
\end{equation}
\textbf{Justificación:} En un escenario de recomendación musical, el espacio de búsqueda es vasto (miles de canciones). El Recall es crucial porque nos indica la capacidad del modelo para "encontrar" las canciones correctas en este pajar, independientemente de su orden exacto. Un alto Recall asegura que el usuario vea opciones relevantes.

\paragraph{NDCG (Normalized Discounted Cumulative Gain)} Esta métrica evalúa no solo si los ítems relevantes están presentes en las recomendaciones, sino también su posición dentro de la lista. Dado que los usuarios tienden a interactuar más con los ítems ubicados en las primeras posiciones, NDCG asigna mayores pesos a los ítems relevantes que aparecen al inicio. Se calcula como:
\begin{equation}
    \text{NDCG@k} = \frac{1}{Z_k} \sum_{i=1}^k \frac{2^{\text{relevancia}_i} - 1}{\log_2(i+1)},
\end{equation}
donde $Z_k$ es un factor de normalización que asegura que el valor máximo de NDCG sea 1.
\textbf{Justificación:} El NDCG es vital para la experiencia de usuario. No basta con recomendar buenas canciones; deben aparecer al principio de la lista (top-k) para maximizar la probabilidad de clic o reproducción. Un NDCG alto refleja un ordenamiento óptimo.

Estas métricas, en conjunto, proporcionan una visión integral del desempeño del modelo, evaluando tanto su capacidad de recuperación (Recall) como la calidad del ordenamiento (NDCG).

Las métricas Recall y NDCG fueron seleccionadas debido a su relevancia en el contexto de los sistemas de recomendación. Estas métricas permiten evaluar tanto la capacidad del modelo para recuperar ítems relevantes como la calidad del ordenamiento de las recomendaciones, aspectos fundamentales para garantizar una experiencia de usuario satisfactoria.

\paragraph{Recall} En sistemas de recomendación, el Recall es particularmente útil para medir la cobertura de los ítems relevantes. Esto es crucial en aplicaciones donde el objetivo principal es maximizar la exposición de los usuarios a contenido relevante, como en plataformas de música o video bajo demanda. Un alto valor de Recall asegura que el sistema no omita ítems importantes para el usuario.

\paragraph{NDCG} Por otro lado, NDCG complementa al Recall al incorporar la posición de los ítems relevantes dentro de la lista de recomendaciones. Dado que los usuarios tienden a interactuar más con los ítems ubicados en las primeras posiciones, NDCG es una métrica esencial para evaluar la calidad del ordenamiento. Esto es especialmente relevante en escenarios donde la atención del usuario es limitada y las recomendaciones deben ser altamente precisas desde el inicio.
%%%%%%
En conjunto, estas métricas proporcionan una evaluación integral del desempeño del modelo, evaluando tanto su capacidad de recuperación como la calidad del ordenamiento de las recomendaciones. Su uso está ampliamente respaldado en la literatura sobre sistemas de recomendación, lo que refuerza su validez en este trabajo.


% Discusión
\section{Discusión}
\label{sec:discusion}

\subsection{Interpretación de Resultados}
Los resultados obtenidos (Recall@10: 0.6225, NDCG@10: 0.5478) validan la eficacia de la arquitectura \textit{Two-Tower} multimodal propuesta. El valor de Recall@10 indica que el modelo logra identificar una proporción significativa de ítems relevantes dentro de las primeras 10 recomendaciones, mitigando el problema de \textit{cold-start} al no depender exclusivamente de interacciones históricas.

Por otro lado, los valores de NDCG destacan la capacidad del modelo para priorizar ítems relevantes en posiciones superiores. Sin embargo, la diferencia entre Recall y NDCG sugiere que, aunque el modelo recupera ítems relevantes, existe margen de mejora en el ordenamiento fino de las recomendaciones. La integración de audio, texto e imágenes enriquece la representación latente, permitiendo inferir similitudes semánticas y estéticas que serían invisibles para modelos unimodales.

\subsection{Comparación y Limitaciones}
A diferencia del filtrado colaborativo tradicional, nuestro enfoque aprovecha el contenido denso. El uso de DeBERTa y ResNet permite capturar la semántica compleja de las canciones. Sin embargo, la complejidad computacional de procesar cuatro modalidades simultáneamente impone restricciones de hardware. Además, la dependencia de metadatos completos (letras, carátulas) puede introducir ruido si estos faltan.


% Conclusiones
\section{Conclusiones y Trabajo Futuro}
\label{sec:conclusiones}

En este trabajo se presentó el diseño, implementación y evaluación de un sistema de recomendación musical multimodal basado en una arquitectura \textit{Two-Tower} con estrategia de \textit{Late Fusion}. El objetivo principal fue mitigar las limitaciones de los enfoques colaborativos tradicionales mediante la integración de información rica de contenido: audio, texto (letras), imágenes (portadas) y metadatos tabulares.

\subsection{Síntesis de Hallazgos}
Los resultados experimentales validan la hipótesis de que la integración de múltiples modalidades permite construir representaciones de ítems más robustas y semánticamente significativas. El modelo alcanzó un Recall@10 de 0.6225 y un NDCG@10 de 0.5478, demostrando una capacidad competente para recuperar y ordenar ítems relevantes en un espacio de búsqueda denso. La estrategia de fusión tardía demostró ser efectiva para combinar embeddings provenientes de codificadores heterogéneos (ResNet-18, mDeBERTa, MLP) sin incurrir en costos computacionales prohibitivos durante el entrenamiento.

\subsection{Contribuciones Principales}
Las contribuciones más destacadas de esta investigación incluyen:
\begin{itemize}
    \item \textbf{Dataset Multimodal Unificado}: la creación de un conjunto de datos curado que vincula interacciones de usuario con tres modalidades de contenido no estructurado (audio, texto, imagen), un recurso valioso para la comunidad de investigación en MIR (\textit{Music Information Retrieval}).
    \item \textbf{Arquitectura Eficiente}: la implementación de una arquitectura modular que utiliza técnicas de eficiencia como LoRA y pre-cómputo de características, alineada con los principios de la metodología TRIZ para resolver la contradicción entre precisión y costo computacional.
    \item \textbf{Validación de Fusión Tardía}: la demostración empírica de que la fusión de características de alto nivel en la etapa final de la torre del ítem es suficiente para capturar la sinergia entre modalidades en el dominio musical.
\end{itemize}

\subsection{Trabajo Futuro}
A pesar de los resultados prometedores, existen varias líneas de investigación abiertas para mejorar el sistema:
\begin{itemize}
    \item \textbf{Mecanismos de Atención Dinámica}: implementar mecanismos de atención a nivel de modalidad (\textit{Modality-level Attention}) que permitan al modelo ponderar dinámicamente la importancia de cada fuente de información para cada canción o usuario específico.
    \item \textbf{Evaluación en Cold-Start Estricto}: realizar pruebas específicas con ítems y usuarios completamente nuevos para cuantificar con mayor precisión la ganancia en escenarios de arranque en frío puro.
    \item \textbf{Escalabilidad del Dataset}: ampliar el conjunto de datos más allá de las 10,000 canciones para evaluar la capacidad de generalización del modelo en catálogos de escala industrial.
    \item \textbf{Inferencia en Tiempo Real}: optimizar el pipeline de inferencia mediante la cuantización de modelos y el uso de bases de datos vectoriales para la recuperación de vecinos más cercanos (ANN).
\end{itemize}

En conclusión, este proyecto sienta las bases para sistemas de recomendación más holísticos que `escuchan', `leen' y `ven' la música, acercándose más a la forma en que los humanos experimentan y descubren el arte.


\bibliography{referencias}

\end{document}